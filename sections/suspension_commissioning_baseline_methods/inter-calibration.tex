\subsection{Inter-calibration}
Inter-calibration is defined by the tasks that matches the calibration between two sensors.
One example where inter-calibration would be useful is sensor correction (to be discussed in Sec.~\ref{sec:sensor_correction}) where the seismometer readout is used to cancel the seismic noise coupling in the relative displacement sensor readouts from the preisolator.
Other usage would be to inter-calibrate the sensor readouts from different stages of a suspension for consistency, diagnosis, sensor fusion, or hierarchical control.

Now, let's say we have two sensor readouts $y_1(t)$ and $y_2(t)$ measuring a common signal $x(t)$.
So, assuming that the sensor dynamics is a constant (calibrated), the readouts read
\begin{equation}
	y_1(t) = a_1\left[x(t) + n_1(t)\right]\,,
\end{equation}
and
\begin{equation}
	y_2(t) = a_2\left[x(t) + n_2(t)\right]\,,
\end{equation}
where $n_1(t)$ and $n_2(t)$ are the sensor noises of sensing readout $y_1(t)$ and $y_2(t)$, and $a_1$ and $a_2$ are constants that represents the calibration error.
We cannot measure the calibration errors directly unless we have a third, well-calibrated sensor.
So, for simplicity, let's assume $a_1=1$, and we only care about relative calibration error.
The goal is to find a constant $k=k^*$, such that $y_1 - k^* y_2 = n_1-n_2$, or in other words, $k^* a_2 = 1$.
To do so, one might be tempted to minimize the cost function
\begin{equation}
	J(k) = \left\langle \left[ y_1(t)-ky_2(t) \right]^2\right\rangle\,,
	\label{eqn:cost_function_inter-calibration}
\end{equation}
for $t={t_1,t_2,\dots,T}$, where $T$ is a very large number, and $\langle \cdot \rangle$ denotes the average.
To see why this might work , we expand Eqn.~\eqref{eqn:cost_function_inter-calibration}
\begin{equation}
	\begin{split}
		\left\langle \left[ y_1(t)-ky_2(t) \right]^2\right\rangle &= \left\langle \left[(1-ka_2)x(t) + n_1(t) - ka_2n_2(t) \right]^2 \right\rangle \\
		&= \left(1-ka_2\right)^2\left\langle x(t)^2 \right\rangle + \left\langle n_1(t)^2 \right\rangle + \left(ka_2\right)^2\left\langle n_2(t)^2 \right\rangle \,,
	\end{split}
	\label{eqn:cost_function_inter-calibration_expanded}
\end{equation}
where we assumed $x(t)$, $n_1(t)$, and $n_2(t)$ are uncorrelated, zero-mean signals so the cross-terms are zero.
Inspecting Eqn.~\eqref{eqn:cost_function_inter-calibration_expanded}, we see that the second term $\left\langle n_1(t)^2 \right\rangle$ is not minimizable so minimizing $J(k)$ is equivalent to minimzing
\begin{equation}
	J'(k) = \left(1-ka_2\right)^2\left\langle x(t)^2 \right\rangle + \left(ka_2\right)^2\left\langle n_2(t)^2 \right\rangle\,,
	\label{eqn:cost_function_inter-calibration_equivalent}
\end{equation} 
which has a minimum of
\begin{equation}
	\min J'(k) = \frac{\left\langle x(t)^2 \right\rangle^2 \left\langle n_2(t)^2\right\rangle + \left\langle x(t)^2 \right\rangle \left\langle n_2(t)^2\right\rangle^2}{\left[\left\langle x(t)^2\right\rangle + \left\langle n_2(t)^2\right\rangle\right]^2}\,,
\end{equation}
at
\begin{equation}
	k = \frac{1}{a_2}\frac{\left\langle x(t)^2 \right\rangle}{\left\langle x(t)^2 \right\rangle + \left\langle n_2(t)^2 \right\rangle}\,,
\end{equation}
which is only correct when the SNR is high, i.e. $\left\langle x(t)^2 \right\rangle \gg \left\langle n_2(t)^2 \right\rangle$.

Typically, the signal $x(t)$ is localized at certain frequencies so it's 
To deal with low SNR signals, we consider the Laplace transform (frequency domain) of the signals instead.
In Laplace domain, the sensor readouts read
\begin{equation}
	Y_1(s) = a_1X(s) + N_1(s)\,,
\end{equation}
and
\begin{equation}
	Y_2(s) = a_2X(s) + N_2(s)\,.
\end{equation}
Again, for simplicity we let $a_1 = 1$ and we want to find $k=k^*=1/a_2$.
We do so by minimizing the cost function
\begin{equation}
	\boxed{
		J_f (k) = \sum_{f} \left\lvert Y_1(s) - kY_2(s)\right\rvert w(f)
	}\ ,
	\label{eqn:cost_function_inter-calibration_frequency_domain}
\end{equation}
where $f$ is frequency, $s=i2\pi f$, $i$ is the imaginary number, $w(f)$ is a weighting function.
Eqn.~\eqref{eqn:cost_function_inter-calibration_frequency_domain} is no different from Eqn.~\eqref{eqn:cost_function_inter-calibration} if the weighing function is equal to one at all frequencies $w(f)=1$.
But, we can choose to weight individual data point a each frequency, and it makes sense to choose $w(f)$ such that the value is high at frequencies where the SNR is high.
A natural candidate of the weighing function would be the coherence function between $y_1(t)$ and $y_2(t)$, so
\begin{equation}
	\boxed{
		w(f) = C_{y_1y_2}(f)\,,
}
\end{equation}
 where $C_{y_1y_2}(f)$ is the coherence function between the sensor readout $y_1(t)$ and $y_2(t)$.
 This is because a high coherence would indicate a high SNR data point, while low coherence would indicate a low SNR data point.
 We can also choose the weighing function to be 1 if the coherence exceeds certain threshold and 0 at other frequencies.
 This effectively masks out the data points with low SNR.
 
Another interesting cost function to consider is
\begin{equation}
	J_\mathrm{log} (k) = \sum_f \log\left\lvert \frac{Y_1(s)}{kY_2(s)} \right\rvert w(f)\,,
\end{equation}
since we often have measurements of the ratios between $Y_1(s)$ and $Y_2(s)$ as a transfer function measurement, but not individual $Y_1(s)$ and $Y_2(s)$.
But here we need to take care of the sign so





