\subsection{Sensor fusion}
Sensor fusion is a technique that combines multiple sensors into a virtual ``super sensor'' that has better performance compared to that of the individual sensors.
In KAGRA, sensor fusion is used mostly for the preisolator, where active seismic isolation is critical.
In particular, LVDTs and Geophones are blended using complementary filters so the displacement readouts at the preisolator benefits from the advantages from the goephones (seismic noise-free and low sensor noise at higher frequencies) and LVDTs (low sensor noise at lower frequencies).
Sensor correction is used to remove the seismic noise coupling from the LVDT readouts at the preisolator using the seismometer readout.
Strictly speaking, sensor correction also utilizes complementary filter.
For optimal performance, sensor correction should be considered as a complementary filter problem.
But, sensor correction was not conventionally considered this way during the first attempt in KAGRA \cite{sensor_correction_gain_tuning, comment_to_sensor_correction_gain_tuning}.
Instead, sensor correction filters were constructed as a high-pass filter using heuristics.
So, we will make a distinction between complementary filter problems and sensor correction filters in this section.

This section is organized as follows.
In Sec.~\ref{sec:complementary_filter}, we will discuss some predefined complementary filters and how to design and optimize the filter parameters.
In Sec.~\ref{sec:sensor_correction}, we will introduce the purpose of sensor correction and we will discuss how was it design and how can we improve it.
In Sec.~\ref{sec:sensor_fusion_examples}, we will provide examples on these complementary and sensor correction filters design.

\subsubsection{Sensor fusion using complementary filter \label{sec:complementary_filter}}

\subsubsection{Sensor correction \label{sec:sensor_correction}}
\subsubsection{Examples \label{sec:sensor_fusion_examples}}