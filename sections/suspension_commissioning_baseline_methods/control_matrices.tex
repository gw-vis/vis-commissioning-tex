\subsection{Control matrices}
In this section, we will discuss how control matrices (sensor matrices and actuation matrices) can be derived, and how to fine tune them (diagonalization).

As is discussed in Sec.~\ref{sec:suspension_commissioning_tasks_further_elaboration}, there are many sensors and actuators in a single stage of suspension.
And in fact, sensors and actuators mostly come in pairs for KAGRA suspensions.
For instance, at the preisolator stage, there are 3 linear variable differential transformers (LVDTs) (not counting inertial sensors), and 3 coil-magnetic actuators placed at close proximity to the LVDTs.
Another example would be the intermediate-mass stage, which has 6 optical sensors and electromagnetic actuator (OSEMs) or photo-reflective displacement sensors (PS).
One exception would be the optics stage which has four coil-magnetic actuators, while only have an optical lever that can measure displacements from 3 degrees of freedom (DoFs).
In principle, the number of independent sensors, or I should say, readouts, decides the number of degrees of freedom that we can measure.
The minimum number of sensors that is needed is at least the number of degrees of freedom that we want to measure.
And, so this is equivalent to saying that, the maximum number of degrees of freedom that we can measure, is the maximum number of sensors.
The same goes to actuators.

Having sufficient number of sensors and actuators at a stage gives us the possibility to sense or actuate the motion at different DoFs of a stage.
But, the sensors and actuators may not be aligned directly to the degrees of freedom that we would like to measure.
For example, the LVDTs at the preisolator stage are located at the outer edge of the preisolator table, which is circular in shape so three LVDTs are sensing the tangential displacement of the preisolator table at three different angles (see \cite{all_of_the_vibration} for a simple figure).
Therefore, the readouts must be transformed from the sensor basis to the basis that we would like to perform control.
The same goes to actuation.
This is done via control matrices.

The desired basis is typically Cartesian (longitudinal, transverse, and vertical) with Euler angles (roll, pitch, yaw).
The Cartesian basis is defined locally for each suspension, where longitudinal direction is typically the normal of the high-reflectivity (HR) side of the optics, vertical is up, and transverse is the direction such that the cross product of the longitudinal direction and transverse direction is vertical.
As for the angles, roll, pitch, and yaw are defined as the angular displacements along the longitudinal, transverse, and vertical axis, respectively.
These DoFs can be locally defined for each stages as well.
For example, we would say IP yaw, IM yaw, and TM yaw, which are yaw at the preisolator, intermediate mass, and optics, respectively.
But, in this section we will focus on stage-wise control matrices, i.e. not mixing sensor readouts from different stages.

In a nutshell, here is how control matrices are derived for each stage.
We express each sensor readout $y_i(t)$ as a superposition of the stage's displacements $x_j(t)$,
\begin{equation}
	y_i(t) = \sum_j^m C_{ij} x_j(t)\,,
\end{equation}
where $i=1,2,3,...,n$ where $n$ is the number of sensors, and $j=1,2,3,...,m$ where $m$ is the number of degrees of freedom, $C_{ij}$ is the matrix elements of a matrix $\mathbf{C}$ and $\mathbf{C}\in \mathbb{R}^{n\times m}$.
$C_{ij}$ is the coupling coefficient of displacements $x_j(t)$ to sensor  $y_i(t)$
Then, the sensing matrix, defined by a conversion that maps the sensor readouts to the stage displacements, is simply
\begin{equation}
	\mathbf{C}_{\mathrm{sensing}} = \mathbf{C}^{-1}\,,
\end{equation}
where $\mathbf{C}_\mathrm{sensing} \in \mathbb{R}^{m\times n}$, such that $\mathbf{x}(t) = \mathbf{C}_\mathrm{sensing}\mathbf{y}(t)$.

The same goes to actuation.
Suppose we have actuation signals $u_i(t)$ injected to the $i^\mathrm{th}$ actuator.
Since the actuators are not purely acting on a single DoF, the actuation signal reads (at DC only, at higher frequencies there's phase shift. To be discussed in Sec.~\ref{sec:control_matrices_frequency_dependent_matrices}),
\begin{equation}
	u_i(t) = \sum_j^m D_{ij} x_j(t)\,,
\end{equation}
where $i=1,2,3,...,n$ where $n$ is the number of actuators, $j=1,2,3,...,m$ where $m$ is the number of DoFs, and $D_{ij}$ is the matrix elements of a matrix $\mathbf{D}$ and $\mathbf{D}\in \mathbb{R}^{n\times m}$.
Here, different from $C_{ij}$, the scale of $D_{ij}$ doesn't not matter, so long as the row vectors $D_{ij}$ has a direction parallel to the actuation.
Again, the actuation matrix, defined by the matrix mapping the actuation to the stage displacements, is simply
\begin{equation}
	\mathbf{D}_\mathrm{actuation} = \mathbf{D}^{-1}\,,
\end{equation}
and $\mathbf{D}_\mathrm{actuation}\in\mathbb{R}^{m\times n}$, such that $\mathbf{x}(t) = \mathbf{D}_\mathrm{actuation}\mathbf{u}(t)$.

Now, initial control matrices can be determined from the location and orientation, i.e. from geometry, of the sensors and actuators with respect to the stage.
\cite{sr_suspension_diagonalization} and \cite{bs_suspension_diagonalization} gives detailed calculation of the actuation and sensing matrices for the SR suspensions and BS suspensions for all stages, except sensing for the optics stage.
The sensing of the optics stage utilizes an optical lever setup and the derivation is more involved.
The derivation of the initial sensing matrix of optical levers are given in \cite{sensing_matrices_oplev}.
Details of these initial control matrices will be be discussed here.

Having the initial matrices may not be sufficient as residual cross-couplings may still remain.
In Sec.~\ref{sec:control_matrices_sensing_matrices}, we will discuss how to modify the sensing matrices so each sensor readout measures one pure DoF.
In Sec.~\ref{sec:control_matrices_actuation_matrices}, we will discuss how to modify the actuation matrices so each actuation signal purely moves the stage in one direction (at one paricular frequency only).
In Sec.~\ref{sec:control_matrices_frequency_dependent_matrices}, we will discuss a more general actuation matrix, which is a transfer function matrix, i.e. frequency dependency.
These methods are referred to sensor and actuation ``diagonalization'' in KAGRA.


\subsubsection{Sensing matrices \label{sec:control_matrices_sensing_matrices}}
Sensor and actuation diagonalization are two separate procedures but sensor diagonalization must happen before actuation diagonalization, as it depends on the sensing readouts.
When sensor diagonalization is performed, the actuators should be re-diagonalized if it was done with a different sensing matrix.
Now, let's say we have raw sensing signals $\mathbf{y}(t)=y_i(t)$, where $i=1,2,3,...,n$, and $n$ is the number of sensors we have.
These signals are what the sensors measure.
They can be raw voltage, or calibrated in some sort of displacement units.
We have a initial sensing matrix $\mathbf{C}_\mathrm{sensing}^\mathrm{initial}$ that we derived from the geometric location and the orientation of the sensors.
Hence, we have some displacement readouts
\begin{equation}
	\mathbf{x}_\mathrm{readout}(t)=\mathbf{C}_\mathrm{sensing}^\mathrm{initial}\mathbf{y}(t)=\mathbf{C}_\mathrm{coupling}\mathbf{x}(t)\approxeq\mathbf{x}(t)\,,
\end{equation}
where $\mathbf{x}(t)$ is the displacements of the suspensions that we would like to measure and $\mathbf{C}_\mathrm{coupling}$ is a coupling matrix, with off-diagonal terms being the cross-coupling coefficients.

If the initial sensing matrix is perfect, then the coupling matrix $\mathbf{C}_\mathrm{coupling}$ will be the identity matrix $\mathbf{I}$ such that $\mathbf{x}_\mathrm{readout}(t)=\mathbf{x}(t)$.
But, in general, this is not the case.

\subsubsection{Actuation matrices \label{sec:control_matrices_actuation_matrices}}
\subsubsection{Frequency dependent matrices \label{sec:control_matrices_frequency_dependent_matrices}}