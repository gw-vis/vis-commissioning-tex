\subsection{Control matrices}
In this section, we will discuss how control matrices (sensor matrices and actuation matrices) can be derived, and how to fine tune them (diagonalization).

As is discussed in Sec.~\ref{sec:suspension_commissioning_tasks_further_elaboration}, there are many sensors and actuators in a single stage of suspension.
And in fact, sensors and actuators mostly come in pairs for KAGRA suspensions.
For instance, at the preisolator stage, there are 3 linear variable differential transformers (LVDTs) (not counting inertial sensors), and 3 coil-magnetic actuators placed at close proximity to the LVDTs.
Another example would be the intermediate-mass stage, which has 6 optical sensors and electromagnetic actuator (OSEMs) or photo-reflective displacement sensors (PS).
One exception would be the optics stage which has four coil-magnetic actuators, while only have an optical lever that can measure displacements from 3 degrees of freedom (DoFs).
In principle, the number of independent sensors, or I should say, readouts, decides the number of degrees of freedom that we can measure.
The minimum number of sensors that is needed is at least the number of degrees of freedom that we want to measure.
And, so this is equivalent to saying that, the maximum number of degrees of freedom that we can measure, is the maximum number of sensors.
The same goes to actuators.

Having sufficient number of sensors and actuators at a stage gives us the possibility to sense or actuate the motion at different DoFs of a stage.
But, the sensors and actuators may not be aligned directly to the degrees of freedom that we would like to measure.
For example, the LVDTs at the preisolator stage are located at the outer edge of the preisolator table, which is circular in shape so three LVDTs are sensing the tangential displacement of the preisolator table at three different angles (see \cite{all_of_the_vibration} for a simple figure).
Therefore, the readouts must be transformed from the sensor basis to the basis that we would like to perform control.
The same goes to actuation.
This is done via control matrices.

The desired basis is typically Cartesian (longitudinal, transverse, and vertical) with Euler angles (roll, pitch, yaw).
The Cartesian basis is defined locally for each suspension, where longitudinal direction is typically the normal of the high-reflectivity (HR) side of the optics, vertical is up, and transverse is the direction such that the cross product of the longitudinal direction and transverse direction is vertical.
As for the angles, roll, pitch, and yaw are defined as the angular displacements along the longitudinal, transverse, and vertical axis, respectively.
These DoFs can be locally defined for each stages as well.
For example, we would say IP yaw, IM yaw, and TM yaw, which are yaw at the preisolator, intermediate mass, and optics, respectively.
But, in this section we will focus on stage-wise control matrices, i.e. not mixing sensor readouts from different stages.

In a nutshell, here is how control matrices are derived for each stage.
We express each sensor readout $y_i(t)$ as a superposition of the stage's displacements $x_j(t)$,
\begin{equation}
	y_i(t) = \sum_j^m C_{ij} x_j(t)\,,
\end{equation}
where $i=1,2,3,...,n$ where $n$ is the number of sensors, and $j=1,2,3,...,m$ where $m$ is the number of degrees of freedom, $C_{ij}$ is the matrix elements of a matrix $\mathbf{C}$ and $\mathbf{C}\in \mathbb{R}^{n\times m}$.
$C_{ij}$ is the coupling coefficient of displacements $x_j(t)$ to sensor  $y_i(t)$
Then, the sensing matrix, defined by a conversion that maps the sensor readouts to the stage displacements, is simply
\begin{equation}
	\mathbf{C}_{\mathrm{sensing}} = \mathbf{C}^{\dagger}\,,
\end{equation}
where $\mathbf{C}_\mathrm{sensing} \in \mathbb{R}^{m\times n}$, such that $\mathbf{x}(t) = \mathbf{C}_\mathrm{sensing}\mathbf{y}(t)$.

The same goes to actuation.
Suppose we have actuation signals $u_i(t)$ injected to the $i^\mathrm{th}$ actuator.
Since the actuators are not purely acting on a single DoF, the actuation signal reads (at DC only, at higher frequencies there's phase shift. To be discussed in Sec.~\ref{sec:control_matrices_frequency_dependent_matrices}),
\begin{equation}
	u_i(t) = \sum_j^m D_{ij} x_j(t)\,,
\end{equation}
where $i=1,2,3,...,n$ where $n$ is the number of actuators, $j=1,2,3,...,m$ where $m$ is the number of DoFs, and $D_{ij}$ is the matrix elements of a matrix $\mathbf{D}$ and $\mathbf{D}\in \mathbb{R}^{n\times m}$.
Here, different from $C_{ij}$, the scale of $D_{ij}$ doesn't not matter, so long as the row vectors $D_{ij}$ has a direction parallel to the actuation.
Again, the actuation matrix, defined by the matrix mapping the actuation to the stage displacements, is simply
\begin{equation}
	\mathbf{D}_\mathrm{actuation} = \mathbf{D}^{\dagger}\,,
\end{equation}
and $\mathbf{D}_\mathrm{actuation}\in\mathbb{R}^{m\times n}$, such that $\mathbf{x}(t) = \mathbf{D}_\mathrm{actuation}\mathbf{u}(t)$.

Now, initial control matrices can be determined from the location and orientation, i.e. from geometry, of the sensors and actuators with respect to the stage.
\cite{sr_suspension_diagonalization} and \cite{bs_suspension_diagonalization} gives detailed calculation of the actuation and sensing matrices for the SR suspensions and BS suspensions for all stages, except sensing for the optics stage.
The sensing of the optics stage utilizes an optical lever setup and the derivation is more involved.
The derivation of the initial sensing matrix of optical levers is given in \cite{sensing_matrices_oplev}.
Details of these initial control matrices will be be discussed here.

Having the initial matrices may not be sufficient as residual cross-couplings may still remain.
In Sec.~\ref{sec:control_matrices_sensing_matrices}, we will discuss how to modify the sensing matrices so each sensor readout measures one pure DoF.
In Sec.~\ref{sec:control_matrices_actuation_matrices}, we will discuss how to modify the actuation matrices so each actuation signal purely moves the stage in one direction (at one paricular frequency only).
In Sec.~\ref{sec:control_matrices_frequency_dependent_matrices}, we will discuss a more general actuation matrix, which is a transfer function matrix, i.e. frequency dependency.
These methods are referred to sensor and actuation ``diagonalization'' in KAGRA.


\subsubsection{Sensing matrices \label{sec:control_matrices_sensing_matrices}}
Sensor and actuation diagonalization are two separate procedures but, as we shall we, sensor diagonalization must happen before actuation diagonalization, as it depends on the sensing readouts.
When sensor diagonalization is performed, the actuators should be re-diagonalized if it was done with a different sensing matrix.
Now, let's say we have raw sensing signals $\mathbf{y}(t)=y_i(t)$, where $i=1,2,3,...,n$, and $n$ is the number of sensors we have.
These signals are what the sensors measure.
They can be raw voltage, or calibrated in some sort of displacement units.
We have a initial sensing matrix $\mathbf{C}_\mathrm{sensing}^\mathrm{initial}$ that we derived from the geometric location and the orientation of the sensors.
Hence, we have some displacement readouts
\begin{equation}
	\mathbf{x}_\mathrm{readout}(t)=\mathbf{C}_\mathrm{sensing}^\mathrm{initial}\mathbf{y}(t)=\mathbf{C}_\mathrm{coupling}\mathbf{x}(t)\approxeq\mathbf{x}(t)\,,
\end{equation}
where $\mathbf{x}(t)$ is the displacements of the suspensions that we would like to measure and $\mathbf{C}_\mathrm{coupling}$ is a coupling matrix, with off-diagonal terms being the cross-coupling coefficients.

If the initial sensing matrix is perfect, then the coupling matrix $\mathbf{C}_\mathrm{coupling}$ will be the identity matrix $\mathbf{I}$ such that $\mathbf{x}_\mathrm{readout}(t)=\mathbf{x}(t)$.
But, in general, this is not the case.
If the off-diagonal elements in the coupling matrix are not zero, then the displacement readouts $\mathbf{x}_\mathrm{readout}(t)$ are cross-coupled, and the goal of diagonalization is to adjust the sensing matrix so individual readout is only coupling to one degree of freedom.
To decouple the readouts, we simply need to modify the sensing matrix such that
\begin{equation}
	\boxed{
		\mathbf{C}_\mathrm{sensing} = \left(\mathbf{C}_\mathrm{coupling}\right)^{-1} \mathbf{C}_\mathrm{sensing}^\mathrm{initial}
}\ .
	\label{eqn:sensing_matrix_diagonalization}
\end{equation}
Depending on the suspensions' real-time model architecture, this can be done by multiplying an addition matrix after the initial sensing matrix, or it can be done by simply modifying the existing matrix.

To measure the coupling matrix $\mathbf{C}_\mathrm{coupling}$, we have to rely on resonance features of the suspension.
Some resonant modes of the suspension only involve a single direction.
If these modes have distinguishable frequencies from other modes, then we can estimate the off-diagonal coupling coefficients ratios between the $i^\mathrm{th}$ readout and $j^\mathrm{th}$ readout,
\begin{equation}
	\boxed{
		C_{\mathrm{coupling}, ij}=
		\begin{cases}
			\left\lvert\frac{X_{\mathrm{readout}, i}(s)}{X_{\mathrm{readout}, j}(s)}\right\rvert_{f=f_j}\,, & \angle\frac{X_{\mathrm{readout}, i}(s)}{X_{\mathrm{readout}, j}(s)}\Big\vert_{f=f_j}\approx 0\,,\\
			-\left\lvert\frac{X_{\mathrm{readout}, i}(s)}{X_{\mathrm{readout}, j}(s)}\right\rvert_{f=f_j}\,, & \angle\frac{X_{\mathrm{readout}, i}(s)}{X_{\mathrm{readout}, j}(s)}\Big\rvert_{f=f_j}\approx \pm \pi\,,\\
		\end{cases}
	}
\end{equation}
where $f_j$ is the resonance frequency of the $x_j$ degree of freedom and $X_{\mathrm{readout},i}(s)$ is the Laplace transform of the $i^\mathrm{th}$ readout of $\mathbf{x}_\mathrm{readout}(t)$.
And note here we have explicitly specify the $s$ dependency to distinguish transfer functions and matrices.

Here, a few assumptions are made.
First of all, assume that the suspension stages are oscillating to the extent the peaks in the spectrum are high enough for us to measure coupling ratios, or to say that the coupling ratios are small.
So, the suspensions must be excite via external actuation, typically using the actuators.
There're two ways this can be done, 1) we can use the actuators to give a ``kick'' to the suspension, i.e. impulse input, so it freely swings in damped oscillation or 2) we can inject white noise actuation to excite the modes.
Mathematically, the resulting spectrum should be the same, as an impulse input is effectively white noise in the frequency domain.
But, the white noise actuation might be easier to implement, as the damped oscillation excited by the impulse may result in low SNR towards steady-state.
Secondly, we assume that the sensors are calibrated such that the diagonal terms of the coupling matrix are ones.
Thirdly, we assume that the coupling ratios measured are purely due to linear readout cross-coupling so, if there's cross-coupling, the ratio between the readouts at resonance frequencies must have a 0 relative or $\pm \pi$ phase.
If the ratios has a phase that is not close to 0 or $\pm \pi$ but the magnitude of the ratios are not small, this would mean that the results are not conclusive so we should not assign the ratio to the coupling matrix.
This could happen due to low SNR, mechanical coupling, or other reasons that are not yet clearly studied.
Unless the reason is known, we should not remove these spurious cross-couplings as it would be overcompensating.

\subsubsection{Actuation matrices \label{sec:control_matrices_actuation_matrices}}
The idea for actuation diagonalization is similar to that of sensor diagonalization but with some minor differences.

Let's say we have raw actuation signals $\mathbf{u}(t)$ and displacements $\mathbf{x}(t)$.
These raw actuation signals are voltage or ``counts'' applied to the actuators, and the $\mathbf{u}(t)$ frame is in the actuator frame, which is not aligned to the displacements $\mathbf{x}$.
We wish to have actuations signals $\mathbf{u}_\mathbf{x}(t)$ that are parallel to the displacements $\mathbf{x}(t)$.
So $\mathbf{u}_\mathbf{x}(t)$ is something like actuation signal in longitudinal, transverse, ..., etc, and this is the variable that the control system manupulates.
Whereas, the raw actuation signal $\mathbf{u}(t)$ is like actuation in first actuator, second actuator, ..., etc.
The actuation signals need to be converted back to the raw actuation signals via multiplying the actuation signals by the actuation matrix, i.e.
\begin{equation}
	\mathbf{u}(t)=\mathbf{D}_\mathrm{actuation}\mathbf{u}_\mathbf{x}(t)\,,
\end{equation}
where $\mathbf{D}_\mathrm{actuation}$ is the actuation matrix.

Now, let's say we have initial actuation matrix $\mathbf{D}_\mathrm{actuation} = \mathbf{D}_\mathrm{actuation}^\mathrm{initial}$.
If the actuation is not diagonalized, the displacement due to actuation signals reads
\begin{equation}
	\mathbf{x}(t) = \mathbf{D}_\mathrm{coupling}\mathbf{u}_\mathbf{x}(t) \,,
	\label{eqn:coupled_displacement_actuation}
\end{equation}
where $\mathrm{D}_\mathrm{coupling}$ is the coupling matrix due to actuation.
The diagonal elements of $\mathbf{D}_\mathrm{coupling}$ are the actuation efficiencies, i.e. the transfer function gain from actuation to displacement.
Here, it would be tempting to multiply the actuation signal by a matrix $\left(\mathbf{D}_\mathrm{coupling}\right)^{-1}$, such that
\begin{equation}
	\mathbf{x}(t)=\mathbf{D}_\mathrm{coupling}\left(\mathbf{D}_\mathrm{coupling}\right)^{-1} \mathbf{u}_\mathbf{x}(t)\,.
\end{equation}
This will diagonalization the actuation, but it is not the ideal approach.
This is because this will make the actuation efficiencies from $\mathbf{u}_\mathbf{x}(t)$ to $\mathbf{x}(t)$ identically ones.
This may cause problems practically as we might want to compare actuation efficiencies or actuation transfer functions from previous measurements.
So instead, it follows that if we multiple the actuation signals $\mathbf{u}_\mathbf{x}(t)$ by $\left(\mathbf{D}_\mathrm{coupling}\right)^{-1}\mathbf{D}_\mathrm{efficiency}$ before converting to the raw actuation signals via the initial actuation matrix  $\mathbf{D}_\mathrm{actuation}^\mathrm{initial}$, then the actuation-displacement relation becomes
\begin{equation}
	\mathbf{x}(t) = \mathbf{D}_\mathrm{coupling}\left[\left(\mathbf{D}_\mathrm{coupling}\right)^{-1}\mathbf{D}_\mathrm{efficiency}\right]\mathbf{u}_\mathbf{x}(t)\,,
	\label{eqn:actuation_displacement_relation_diagonalized}
\end{equation}
where $\mathbf{D}_\mathrm{efficiency}$ is a diagonal matrix with elements being the actuation efficiencies at which the coupling matrix was measured.
The actuation efficiency matrix $\mathbf{D}_\mathrm{efficiency}$ is a free parameter that we can set.
So, for initial setup of the suspensions, this can be any arbitrary desired values.
However, for diagonalization, it's better to maintain the static gain of the actuation transfer functions, i.e. maintaining the previous actuation efficiencies.
This is equivalent to saying that we should set the actuation efficiency matrix to the diagonal elements of the actuation coupling matrix $\mathbf{D}_\mathrm{coupling}$
\begin{equation}
	\mathbf{D}_\mathrm{efficiency} = \text{diag}\left(\mathbf{D}_\mathrm{coupling}\right)\,.
\end{equation}

Now, coming back to the actuation matrix $\mathbf{D}_\mathrm{actuation}$.
If we must modify the actuation matrix for diagonalization (alternatively we can have matrices in between actuation signals and the initial actuation matrix in the system), then Eqn.~\eqref{eqn:actuation_displacement_relation_diagonalized} is equivalent to saying that we should modify the actuation matrix from the initial actuation matrix $\mathbf{D}_\mathrm{actuation}^\mathrm{initial}$ to
\begin{equation}
	\boxed{
		\mathbf{D}_\mathrm{actuation} = \mathbf{D}_\mathrm{actuation}^\mathrm{initial} \left(\mathbf{D}_\mathrm{coupling}\right)^{-1} \mathbf{D}_\mathrm{efficiency}
	}\ .
	\label{eqn:actuation_matrix_diagonalized}
\end{equation}

To measure the actuation coupling, we need to measure the displacement $\mathbf{x}(t)$ due to actuation $\mathbf{u}_\mathbf{x}(t)$.
However, to measure $\mathbf{x}(t)$, we can only rely on the sensing readout $\mathbf{x}_\mathrm{readout}(t)$ from Sec.~\ref{sec:control_matrices_sensing_matrices}.
Therefore, it is important to complete sensor diagonalization before actuation diagnalization such that $\mathbf{x}_\mathrm{readout}(t)=\mathbf{x}(t)$
Also, note that the level of diagonalization that actuation can achieve is at best the level of diagonalization of the sensors.
To measure the actuation coupling matrix, we need to inject actuation signals, typically in the form of a DC offset or a sinusoidal single-frequency line injection.
DC offset inject is good because it would mean that there's no relative-phase between actuation and displacement.
If we choose to use a line injection, we must choose frequency at which both diagonal transfer functions and cross-transfer functions has $0$ or $\pm\pi$ phase.
This is tricky because we can't measure the transfer function unless we diagonalize the actuation!
Another way to approach this is to choose a frequency $f_\mathrm{inject}$ such that the measured $\frac{\mathbf{X}(s)}{\mathbf{U}_\mathbf{x}(s)}\big\rvert_{f=f_\mathrm{inject}}$ has $0$ or $\pm\pi$ phase.
So, the actuation coupling matrix can be measured as
\begin{equation}
	\boxed{
		D_{\mathrm{coupling}, ij} =
		\begin{cases}
			\left\lvert\frac{X_{i}(s)}{U_{\mathbf{x}, j}(s)}\right\rvert_{f=f_\mathrm{inject}}\,, & \angle\frac{X_{j}(s)}{U_{\mathbf{x}, j}(s)}\Big\rvert_{f=f_\mathrm{inject}} \approx \angle\frac{X_{i}(s)}{U_{\mathbf{x}, i}(s)}\Big\rvert_{f=f_\mathrm{inject}}\,,\\
			-\left\lvert\frac{X_{j}(s)}{U_{\mathbf{x}, j}(s)}\right\rvert_{f=f_\mathrm{inject}}\,, & \angle\frac{X_{j}(s)}{U_{\mathbf{x}, j}(s)}\Big\rvert_{f=f_\mathrm{inject}} \approx \angle -\frac{X_{i}(s)}{U_{\mathbf{x}, i}(s)}\Big\rvert_{f=f_\mathrm{inject}}\,,
		\end{cases}	
}
\end{equation}
where $i,j = 1,2,3,...,n$ and $n$ is the degrees of freedom.

Here, there's a few caveats.
First of all, what we are doing here is \textbf{not} actually actuation diagonalization, instead, what we are trying to do is to align the actuators to the suspension stage's degrees of freedom.
But, this alignment has frequency dependency, if the cross-transfer functions are not zero, which is generally true.
This means what we cannot minimize actuation cross-coupling at all frequencies, except at a signal frequency, which is actually not that useful.
The good thing is that actuation diagonalization is not that important for local damping control, if controls are all engaged for all DoFs.
Actuation cross-coupling will be treated as disturbance from one DoF to another, and will be suppressed by active control.
However, actuation coupling will become important when suspension control is not engaged, such as during interferometer locking or observation.
Therefore, we need some way to actually diagnalize the actuations and this will be discussed in Sec.~\ref{sec:control_matrices_frequency_dependent_matrices}.

\subsubsection{Frequency dependent matrices \label{sec:control_matrices_frequency_dependent_matrices}}
As is discussed in Sec.~\ref{sec:control_matrices_actuation_matrices}, a scalar-valued actuation matrix is not useful for diagonalizing the actuators.
With a scalar actuation matrix, the actuation can only be diagonalized at a single frequency so actuation signals in one DoF at other frequencies can still other DoFs.
To correctly model the actuation system, we have to consider transfer function matrix, with input $\mathbf{U}_\mathbf{x}(s)=\left[U_{x_1}(s), U_{x_2}(s),...,U_{x_n}(s)\right]^T$ and output (displacement) $\mathbf{X}(s)=[X_1(s), X_2(s),..., X_n(s)]^T$, where $n$ is the number of  DoFs.
The input-output relation can be written as
\begin{equation}
	\mathbf{X}(s) = \mathbf{P}(s)\mathbf{U}_\mathbf{x}(s)\,,
\end{equation}
where the actuation plant $\mathbf{P}(s)$, i.e. the transfer function matrix is
\begin{equation}
	\mathbf{P}(s)=
	\begin{bmatrix}
		P_{11}(s) & P_{12}(s) & \dots & P_{1n}(s)\\
		P_{21}(s) & P_{22}(s) & \dots & P_{2n}(s)\\
		\dots & \dots & \ddots & \vdots\\
		P_{n1}(s) & P_{n2}(s) & \dots & P_{nn}(s)
	\end{bmatrix}\,,
\end{equation}
where $P_{ij}(s)$ is the SISO transfer function from the $j^\mathrm{th}$ actuation $U_j(s)$ to $i^\mathrm{th}$ displacement $X_i(s)$, and note that $P$/$\mathbf{P}$ here means plant/path/transfer function, and \textbf{not} power spectral density.
$\mathbf{P}(s)$ is the plant of the suspension and cannot be changed. 
The idea of diagonalization is to multiply the actuation signals by a transfer function matrix, i.e. ``the'' actuation matrix, such that the resultant off-diagonal transfer functions (cross-transfer functions) are zero.
Mathematically, we want to design an actuation matrix $\mathbf{D}_\mathrm{actuation}(s)$, such that
\begin{equation}
	\mathbf{X}(s) = \mathbf{P}(s)\mathbf{D}_\mathrm{actuation}(s)\mathbf{U}_\mathbf{x}(s)\,,
\end{equation}
and $\mathbf{P}(s)\mathbf{D}_\mathrm{actuation}(s)$ is a diagonal transfer function matrix, preferably with diagonal entries $P_{ii}(s)$.
It follows that if we set
\begin{equation}
	\mathbf{D}_\mathrm{actuation}(s) = \mathbf{P}(s)^{-1}\text{diag}\left(\mathbf{P}(s)\right)\,,
	\label{eqn:frequency_dependent_actuation_matrix}
\end{equation}
then we can obtain
\begin{equation}
	\begin{bmatrix}
		X_1(s)\\
		X_2(s)\\
		\vdots\\
		X_n(s)
	\end{bmatrix}
	=
	\begin{bmatrix}
		P_{11}(s) &  &  & \\
		& P_{22}(s) &  & \\
		&&\ddots&\\
		&&&P_{nn}(s)
	\end{bmatrix}
	\begin{bmatrix}
		U_{x_1}(s)\\
		U_{x_2}(s)\\
		\vdots\\
		U_{x_n}(s)
	\end{bmatrix}
	\,,
\end{equation}
which is exactly what we want, a diagonalized actuation plant.

There are at least ways to obtain the actuation matrix using Eqn.~\eqref{eqn:frequency_dependent_actuation_matrix}.
The first way is to measure the transfer functions $P_{ij}(s)$ as a complex-valued frequency series and then use a transfer function model to fit it using methods discussed in Sec.~\ref{sec:transfer_function_modeling}.
With the analytical form of the transfer functions, we can construct the transfer function matrix by evaluating Eqn.~\eqref{eqn:frequency_dependent_actuation_matrix}.
The calculation of the inverse of a transfer function matrix is rather involved as it's not easy to obtain the analytical form of the inverse.
If the matrix is small, we can easily express individual elements of the inverse transfer function matrix as a combination of the elements of the transfer function matrix.
However, this becomes tedious as the number of DoFs exceed 3,
This brings us to the second way of obtaining the actuation matrix.
Instead of modeling the individual transfer functions $P_{ij}(s)$, we take the frequency series of the transfer functions and then compute the inverse of the transfer function matrix at each frequency so we can obtain the actuation matrix as a complex-valued frequency series using Eqn.~\eqref{eqn:frequency_dependent_actuation_matrix}.
From here, we can use transfer functions models to fit the individual elements $D_{ij}(s)$ using methods from Sec.~\ref{sec:transfer_function_modeling}. 

\subsubsection{Examples}
Pending
\paragraph{Sensor diagonalization}
\paragraph{Actuation diagonalization}
\paragraph{Frequency depended diagonalization}
